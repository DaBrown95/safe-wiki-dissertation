\chapter{Evaluation}

\section{Ownership of Data}

As discussed in Section \ref{sec:ownership-of-data}, Ownership of Data is an important aspect of the SAFE Network. Access to the \textit{Data Map} of data is what grants access to what is being stored on the network, revoke access or ``forget'' the credentials to the \textit{Data Map} and that data is irrecoverable. Users `own' the access to their data and nobody else does. This is what is so special about the architecture of the SAFE Network and what gives it unique opportunities for software development. Ownership of Data specifically means that you can give people access to data whilst retaining the sole right to its mutation. Think of a public blog for example. The creator, the account used to create the blog, `owns' the ability to edit the blog. They can mutate data as they please as they have access to the keys necessary. Other people can access the data if and only if the `owner' of the blog says they can. Thus being able to host websites on the SAFE Network like this is extremely powerful.

\section{Opportunistic Caching}

 An important feature of \textit{vaults} that hasn't been discussed yet is that of \textit{caching}. In short, if \textit{vaults} keep passing the same chunk of data onto other \textit{vaults} (as a client somewhere retrieves it) they start to cache the chunks. This means in a loose sense that as more and more people start to access data, say a news website, that access gets faster and faster. Through the architecture of the SAFE Network such websites cannot 'go down' due to high traffic, the network simply responds by allowing more \textit{vaults} to cache and serve the data. This property is important to small websites. If a smaller news site suddenly gets high amounts of traffic, for instance a viral article, they might suffer from outages due to high traffic on a central server in the traditional model of the internet. Hosted on the SAFE Network however, the network will respond (autonomously) to the higher volume of traffic and dynamically increase its potential to serve that content.
 
 Another benefit of this caching is geographical proximity. A good example is that of a website that has support for multiple languages. Due to \textit{vaults} caching data, the chunks that correspond to particular languages will be cached in the \textit{vaults} nearest where they are being served. For instance, the chunks that correspond to Japanese content would more likely to be cached near \textit{vaults} in and around Japan as that is where they are being read from the most. Unlike traditional models of client-server, \textit{vaults} caching data means that as more clients access content it gets faster and faster to retrieve that data.
 
 This caching applies to all data, including ZIM files. This means that when something like Wikipedia is uploaded to the network, the more people that access it the faster it gets for everyone. For a popular resource like Wikipedia this can be really important in some areas of the world. For places that have slow internet connections, \textit{vaults} that are close by can help to ensure data can be retrieved by clients as quickly as possible. 
 
 \section{Cost Benefit}
 
 A huge benefit to anyone wanting to build a website or application is how much it costs to store data on the SAFE Network. Traditionally on the internet, you would need to pay for a server and also a domain name to be used. Depending on how much traffic the website gets you would need a more powerful server which costs more money. This is not the case on the SAFE Network. The only cost is in writing data to the network, thats it. Meaning if you wanted to publish a website you would use \textit{Safecoin} to pay for the storage needed, after that however your website is available forever. You don't need to pay anymore money except when you want to update or add new content.
 
 For SAFE Wiki, this means that once a user pays to upload something like Wikipedia (``only'' ~75GB in size) it is then available to everyone for free and forever. It doesn't cost anymore money than it cost to store the file originally. This means that it is possible for not only organisations like Wikimedia to upload files, but also good samaritans using the network. As the ZIM file is stored on the network as immutable data it cannot be deleted or altered. Meaning once uploaded that file is available for consumption by all and forever.

\section{Privacy and Anonymity}

Anonymity and Privacy are not mutually exclusive. Anonymity means \textit{``Nameless; of unknown name; also, of unknown or unavowed authorship''}\cite{anonymous}. True anonymity is very important for many people around the world, a good example being whistleblowers and the work of journalists in adversarial conditions. The identities of individuals is what matters here. Being able to convey a message and be ``nameless''. 

Privacy means \textit{``The state of being in retirement from the company or observation of others; seclusion''}\cite{privacy}. Privacy is important aspect of our lives that some hold to a higher virtue than others. Most would agree however that you should be able to send a message/letter/email to someone and have the contents of that message be private.

To relate the two concepts together in terms of the SAFE Network, I would say that anonymity relates to meta-data and privacy relates to data itself. A real world example of meta-data is something like an address. When you send a letter to someone, the postal service can see the address on the envelope. The letter is not anonymous because the recipient is not ``nameless''. The contents of the envelope, the data, is however private. It is \textit{``in retirement from the company or observation of others''}. The SAFE Network provides guarantees of privacy. Data on that is stored on the SAFE Network has gone through \textit{self encryption} meaning it is nearly impossible for anyone to deduce what that data is. I could upload pictures and documents with the assurance that only I can access them, it is private data. What the SAFE Network does not provide is anonymity in all network activity and that is down to meta-data. When a client requests a chunk of data from a vault, that vault knows the \textit{account} that is requesting the data. However difficult it may be to tie an \textit{account} to an individual, that information exists for a period of time and hence the request is no longer anonymous.

This has big implications for the use case of the network but the nuances of how this impacts things is quite subtle. Data that is stored on the network is ``garbage data'', \textit{vaults} have no idea what the data they are storing means. \textit{Data Maps} are what gives meaning and relationship to chunks of data. At the storage level, chunks have absolutely no discernible relationship to each other apart form their proximity in 256-Bit address space. The access to \textit{Data Maps} is what gives users privacy. If you are the only person who knows where it is and how to decrypt it then the data it leads to is truly private. Once data is stored on the network, it could be said to be anonymous. Nothing publicly available exists to tie it to an individual. When the data is interacted with, true and complete anonymity is broken. \textit{Vaults} know exactly what account is requesting chunks of data. It is thus for the above reasons I propose that the SAFE Network allows pseudo-anonymised interactions. In practically users can be relatively assured that their interaction is anonymous, they must however know that it is non ensured.

Privacy and anonymity is very important for SAFE Wiki. In some parts of the world people do not have the freedom to access any information they want. A modern day example of this is the censorship is that the Chinese government are still censoring access to information about Tiananmen Square\cite{tiananmen-square}. Thus the privacy and pseudonymization of activity is very important. The SAFE Network facilitates the uncensored access to information, which as discussed at the beginning of this report is crucial to developing societies.

\section{Immutability of Data}

As mentioned in Section \ref{subsec:immutability-of-data}, data on the SAFE Network is immutable. Once a chunk is written to the network that chunk of data can never be removed. It is the access to data through \textit{Data Maps} and encryption keys that orchestrates the usage of it. What this means in practical situations is that once someone has access to data, they have access to it forever. There is no way to revoke that right. For SAFE Wiki, this is brilliant. Once a ZIM file is uploaded then a user can access that information forever.

What is troubling however is sometimes a user may indeed wish to delete data permanently. To truly have it be gone forever and not stored by the network. This may be on a personal level or asking a company to delete their personal records. Although not implemented yet, there is talk and an RFC\cite{delete-data-rfc} to bring deletable Immutable Data to the network. However to allow ``Immutable Data'' to be deleted is an interesting concept, I don't know if you could really call it ``Immutable'' if you can indeed delete it.

\section{SAFE Wiki}

What I set out to achieve with SAFE Wiki has been achieved. In its ``final'' state, users can upload ZIM files to the SAFE Network and browse them. I personally feel that the idea behind SAFE Wiki is an extremely viable use case of the SAFE Network. Hosting archives of websites when the user-base does not exist to support them makes sense.

\section{Future Work}

Future work for SAFE Wiki could take one of two approaches. Development could continue on SAFE Wiki itself or another approach could be taken, using the lessons learned from developing SAFE Wiki.

\subsection{ZIM Uploader}

The first step would be to create a new application called ``ZIM Uploader''. This applications only purpose would be to facilitate the management of ZIM files on the SAFE Network. This means that the average user that has no intention of uploading their own ZIM file doesn't need to see this piece of functionality. As this application only serves one purpose it would be far easier to maintain than SAFE Wiki itself.

Indeed it is highly likely that other developers will create applications to facilitate NFS usage on the SAFE Network. As long as they allow users to create a ZIM Folder and place files within them using NFS emulation, then they could be used to manage ZIM files on the network. Remember there really isn't anything special about a ZIM Folder, it is just a MD structure that is emulated through NFS to store ZIM files. If this happens then the maintenance of a ``ZIM Uploader'' would no longer be required.

\subsection{Kiwix JS Extension}
\label{subsec:kiwix-js-safe}

Kiwix JS itself is a browser extension. This means that after installation you have the ability to read ZIM files from local storage, Without needing an internet connection. Kiwix JS simply uses the browser as its run-time environment. Forking away from this approach perhaps fragments things more than they need to be. I propose that instead of pulling Kiwix JS into a desktop application, that through the use of the \textit{Web API} the functionality of SAFE Wiki could be brought to the extension.  With this approach the user would have Kiwix JS installed inside their browser (SAFE Browser or Peruse) and be able to use Kiwix JS as normal. Within a SAFE Network environment however, Kiwix JS would then have the ability to read ZIM files from the network.

Extending Kiwix JS functionality instead of forking it brings many benefits. The first and foremost is that there is the possibility of this added functionality being merged into the main branch of Kiwix JS. Meaning it would be a supported solution provided by Kiwix. This would not only increase the awareness of ``SAFE Wiki'' but means that improvements (bug fixes etc) can be made to one single repository instead of constantly pulling changes across the two streams. As it would only have the functionality to read ZIM files, the changes required are minimal when compared to all the changes that happened inside SAFE Wiki. This approach makes sense, SAFE Wiki simply facilitates another storage medium for Kiwix JS to use.

\subsection{Website}

Through the \textit{Web API} it would be possible to build a website that would facilitate the reading of ZIM files. This means a user could simply visit the website through a SAFE Network compatible browser and then browse ZIM files hosted on the network as they wish. This approach would deliver the internals of Kiwix JS through the browser to run on the client, providing users with an extremely easy method to access the functionality of SAFE Wiki. This approach however does have its drawbacks. It would mean the maintenance of another code base meaning the benefits described in Section \ref{subsec:kiwix-js-safe} would not apply.

\subsection{Suggestion}

The big downside with choosing to build a website would be that users lose the ability to view ZIM files that they have stored locally. I can envision that in the future it would be possible for a user to download ZIM files from the SAFE Network. Thus maintaining the ability to browse the files locally is a key piece of functionality that shouldn't be lost. This means that I suggest the approach outlined in Section \ref{subsec:kiwix-js-safe} is the most suitable. It is purely Kiwix JS, a well established project, with the added ability to read files from the SAFE Network.







